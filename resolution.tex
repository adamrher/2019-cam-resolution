% qjrms4doc.tex V1.10, 4 October 2013

\documentclass[times]{qjrms4}
\usepackage[colorlinks,bookmarksopen,bookmarksnumbered,citecolor=red,urlcolor=red]{hyperref}
\usepackage{moreverb}

%\def\volumeyear{2013}
\def\volumenumber{00}

\begin{document}

\runningheads{A.~R.~Herrington and K.~A.~Reed}{CAM resolution sensitivity}

\title{Parameterized convection, grid-scale clouds and resolution sensitivity in the Community Atmosphere Model}

\author{Adam R. Herrington\corrauth \& Kevin A. Reed}
\address{School of Marine and Atmospheric Sciences, Stony Brook University, Stony Brook, NY 11794}

\corraddr{\url{adam.herrington@stonybrook.edu}}

\begin{abstract}
This paper describes...
\end{abstract}

\keywords{Climate models, physical parameterizations, resolution sensitivity}

\maketitle

\section{Introduction}

An increasing number of Atmospheric General Circulation Models (AGCMs) are being developed to maximize efficiency on massively parallel systems, permitting regionally-refined high-resolution, or even globally high-resolution weather ($\Delta x \leq 5$ km) and climate ($\Delta x \leq 50$ km) simulations \citep{SMTMN2008JCP,MPASatm,Z2014QJRMS,HETAL2016JCLIM,DCMIP16,LetAl2018JAMES}. These models are built using unstructured meshes that while allows for substantial grid flexibility, would require physical parameterizations ({\em{physics}}) that behave consistently as the truncation scale of the model changes with different grid resolutions, referred to as scale-aware physics. The most common approach towards developing scale-aware physics is through the lens of limited area, large-eddy simulations \citep[e.g.,][]{PC2008JAS,AW2013JAS,SZ2018JCLIM}. Through subsequently filtering large-eddy solutions to lower-resolution grids, a relationship between first- and higher-order moments \citep{G1992JFM} may be understood and ultimately parameterized as a function of grid resolution. While this approach is likely necessary for developing scale-aware physics, it is not sufficient. The equations of motions have inherent scale dependencies, with the properties of dynamical modes a function of native grid resolution \citep{O1981JAS,WETAL1997MWR,PG2006JAS,JR2016QJRMS}. Scale-aware physics should also recognize these native grid dependencies.

The sensitivity of the Community Atmosphere Model \citep[CAM;][]{CAM5}, and its predecessor, the Community Climate Model (CCM) to resolution ({\em{resolution}} refers to {\em{horizontal resolution}}, hereafter) is well documented through convergence studies \citep{KW1991JGR,WETAL1995CD,W2008TELLUS,RETAL2013JCLIM,ZetAl2014JCb,HR2017JCLIM}. {\color{red}{CAM/CCM is a well supported climate model,}} but despite thirty years of continual model development, there are robust sensitivities to resolution that have persisted in all versions of the model. This study argues that a unifying cause, the inherent scale sensitivities of the underlying dynamical equations, can explain the robust responses to resolution that occur in CAM/CCM, {\color{red}{since it is difficult to conceive that inevitable responses to native grid resolution could be ignored in the pursuit of scale-aware physics.}}

In CAM/CCM, the atmosphere progressively dries with increasing resolution, seen through a reduction in simulated total precipitable water \citep{KW1991JGR,WETAL1995CD,W2008TELLUS,RETAL2013JCLIM,ZetAl2014JCb,HR2017JCLIM}, which typically, but not always \citep[see][]{WETAL1995CD,ZetAl2014JCb}, coincides with a reduction in cloud cover. \cite{KW1991JGR} and \cite{WETAL1995CD} suggested that the drying of the atmosphere is due to greater magnitude resolved vertical velocities with increasing resolution, with greater subsiding motion increasing the export of dry air from the upper troposphere. This mechanism is consistent with an analysis of moisture budgets in CAM, version 4 \citep[CAM4;][]{CAM4} across multiple resolutions \citep{YETAL2014JCLIM,HR2017JCLIM}.

It is well known that the magnitude of vertical velocities increase with resolution in atmospheric models. While the cause of this sensitivity has been established for large-eddy simulations \citep[see][and references therein]{J2017JAMES}, only recently has the vertical velocity field in AGCMs and their sensitivity to resolution received attention \citep{DETALA2016ACP,OETAL2016JAMES}, albeit with conflicting explanations \citep{RETAL2016CD,HR2018JAMES}. To generalize the relationship between vertical velocity and resolution, let $\alpha$ refer to the ratio of $W_0$, the vertical velocity scale of some reference grid spacing $\Delta x_0$, to $W$, the vertical velocity scale of any $\Delta x$. A power law for $\alpha^{-1}$ in $\Delta x$ is then,
\begin{equation}
\alpha^{-1} = \frac{W}{W_0} = \left( \frac{\Delta x}{\Delta x_0} \right)^n, \label{eq:alpha}
\end{equation}
where $n$ is the power law exponent. 

\cite{RETAL2016CD} derive an estimate $n= b-1$ by combining a scale analysis of the continuity equation with a power law representation $\Delta x^{2b}$ of the second-order structure function of the horizontal wind. Strictly speaking, $\Delta x$ here refers to the distance between two points for which the velocity increment is computed in the structure function, but with this distance set to the model grid-spacing. Observations indicate that $b=\frac{1}{3}$ for scales less than about $1000$ km \citep{CETAL1999JGR}, which according to the Weiner–Khinchin theorem $- \left( 2b+1 \right) = -\frac{5}{3}$ is equal to the slope of the kinetic energy spectrum, and supported by observations of mesoscale flow \citep{NG1985JAS}. \cite{RETAL2016CD} argue that the $-\frac{5}{3}$ slope being common in both observations and models provides an emergent constraint for $b=\frac{1}{3}$ and $n= -\frac{2}{3}$.

In large-eddy simulations, the sensitivity of vertical velocities to resolution is adequately explained by a scale analysis of the dynamical equations \citep{WETAL1997MWR,PG2006JAS,JR2016QJRMS}. For hydrostatic scales relevant to AGCMs, a scale analysis of the Poisson equation gives $W \propto D^{-1}$, where $D$ is the horizontal scale of a buoyancy perturbation driving vertical motion \citep{HR2018JAMES}. In CAM aqua-planet simulations, the largest source of buoyancy is from grid-scale cloud formation, whose horizontal extents are set by the effective resolution of the model (i.e., some multiple of $\Delta x$), indicating $n=-1$ \citep{HR2018JAMES}. \cite{HR2017JCLIM} has shown that the $n=-1$ scaling does not explain the behavior of CAM4 in a convergence experiment, but follow-up studies \citep{HR2018JAMES,HETAL2019JAMES} indicate that the inadequacy of the $n=-1$ scaling is not definitive, due to time-truncation errors associated with fixing the physics time-step ($\Delta t_{phys}$) across resolutions in that study.

Another robust response of the CAM/CCM lineage to resolution is an increase in grid-scale precipitation rates at the expense of parameterized convective precipitation rates. Grid-scale precipitation is formed from grid-scale clouds, but is often referred to in the literature as stratiform precipitaiton and stratiform clouds, and used hereafter. The behavior of the difference precipitation routines with resolution is shown in Figure~\ref{fig:cam-history}, which is a bar-graph of the climatological, global mean stratiform and convective precipitation rates in prior CAM/CCM convergence studies. The propensity for precipitation rates to shift from the convection scheme to the stratiform scheme with resolution has been documented in other AGCMs as well \citep{PS2002CD,TETAL2018CD}, but none have given a satisfactory explanation for this sensitivity. The studies of \cite{KW1991JGR}, \cite{WETAL1995CD} and \cite{W2013QJRMS} indicate that the tendency to reduce $\Delta t_{phys}$ with resolution would by itself reduce the convective precipitation rates, however Figure~\ref{fig:cam-history} (top row) indicates that convergence studies with fixed $\Delta t_{phys}$ still show a reduction in convective precipitation rates with resolution.

\begin{figure}[t]
\begin{center}
\noindent\includegraphics[width=20pc,angle=0]{figs/cam-history.pdf}\\
\end{center}
\caption{Bar-graph of the convective (solid) and stratiform (white) climatological precipitation rates in prior CAM/CCM convergence studies. Each window contains a single convergence study, with identical x-axis; the approximate grid resolution. Colors indicate the model configuration; January ensemble (black) and aqua-planet configurations with SST profiles $QOBS$ (blue) and $CNTL$ (red) after \cite{NH2000ASL}. Studies included in this figure are \cite{KW1991JGR} (CCM1), \cite{WETAL1995CD} (CCM2), \cite{W2008TELLUS} (CAM3), \cite{RETAL2013JCLIM,ZetAl2014JCb,HR2017JCLIM} (CAM4), \cite{ZetAl2014JCb} (CAM5) and this study (CAM6). CCM2* refers to the modified parameter experiment of \cite{WETAL1995CD}, where parameters vary with resolution to reduce the dependence of cloud fraction on resolution.}
\label{fig:cam-history}
\end{figure}

In this study, a convergence experiment using CAM, version 6 (CAM6; \url{https://ncar.github.io/CAM/doc/build/html/users_guide/index.html}) is carried out and analyzed in detail. It is shown that the resolution sensitivity of vertical velocities are well described with $n=-1$ in equation~\eqref{eq:alpha}, provided $\Delta t_{phys}$ is defined in a way that avoids large truncation errors across resolutions. The reduction in convective precipitation rates with resolution in CAM6 is shown to result from the greater magnitude subsiding motion, creating a more stable atmosphere in which the criterion for parameterized convection occurs less often. The feedback of the resolved vertical motion on the physics indicates that the root cause of resolution sensitivity in CAM arises from the sensitivity of resolved dynamical modes to native grid resolution. Section 2 describes CAM6 and details the convergence experiment. Section 3 contains a thorough analysis of the CAM6 simulations and Section 4 provides some discussion and conclusions.

\section{Methods}

\subsection{Dynamical Core}

This study uses the spectral-element dynamical core option of Community Atmosphere Model \citep[CAM-SE;][]{DetAl2012IJHPCA}, coupled with a mass conserving, semi-Lagrangian advection method for accelerated multi-tracer transport \citep[CSLAM;][]{LTOUNGK2017MWR}, and dry-mass vertical coordinate with comprehensive treatment of moisture and energy \citep{LetAl2018JAMES}. The dry dynamics are solved using the high-order, momentum, mass and energy conserving spectral element method \citep{TF2010JCP}, with the elements defined by a cubed-sphere grid. The notation for the horizontal grid resolution is an `$ne$' followed by the number of elements making up an edge of one cubed-sphere face, e.g., $ne30$. Hyper-viscous $\nabla^{4}$ explicit numerical dissipation is applied to temperature, dry pressure thickness, rotational and divergent winds \citep{LetAl2018JAMES}. CSLAM tracer transport uses a finite volume grid constructed from the cubed-sphere of elements, and contains the same degrees of freedom as the dry dynamics.

\subsection{Physical Parameterizations}

The physics are evaluated on the finite-volume CSLAM grid, and the tendencies mapped back to the spectral element grid. The coupled system, referred to as CAM-SE-CSLAM, conserves energy, mass and preserves linear correlations between two reactive species to within machine precision \citep{HL2018MWR}. A coarser physics grid, containing $\frac{5}{9}$ fewer degrees of freedom than the dynamical core grid is also available as part of the CAM-SE-CSLAM package \citep{HETAL2019JAMES}. This lower-resolution physics grid is used in this study, but only as a member of a perturbed parameter ensemble and not in the default convergence experiment. The dynamics time-step is subcycled within a longer physics time-step $\Delta t_{phys}$, and the temperature and momentum increments from the physics are divided by the number of subcycles and added to the dynamical core at the beginning of each subcycle. The full moisture increment from the physics is applied only at the start of the first subcycle to conserve tracer mass \citep[$ftype=2$ option in][]{LW2019JAMES}.

The simulations use the CAM6 physics package. The Cloud Layers Unified by Binormals \citep[CLUBB][]{GETAL2002JAS,BOG2013JCLIM} is an assumed probability distribution function (PDF) high-order closure model that handles shallow convection, planetary boundary layer mixing and cloud macrophysics. The macrophyiscs are coupled with a two-moment bulk cloud microphysics scheme with prognostic precipitation \citep{MG2}, and microphysics are coupled with a three mode Modular Aerosol Model \citep{MAM}. The combined macrophysics/microphysics routines generate stratiform precipitation from stratiform clouds. Deep convection is parameterized using a quasi-equilibrium mass flux scheme \citep{ZM1995AO} and an entraining plume model \citep[referred to as the dilute convective available potential energy, or {\em{dilute CAPE}} hereafter;][]{RB1992JAS, NRJ2008JC} is used as a convective trigger (convection occurs if dilute CAPE $\geq 70$ J/kg), and for closing the mass fluxes in the cloud ensemble. The deep convection scheme also parameterizes convective momentum transport \citep{RR2008JC}.

\subsection{Experimental Design}
 
The convergence experiment is performed in an aqua-planet configuration \citep{NH2000ASL,MWO2016JAMES}, an all ocean planet with fixed, zonally symmetric sea surface temperatures modeled after present day Earth \citep[$QOBS$ in][]{NH2000ASL}. The aqua-planets are in a perpetual equinox, and aerosols are largely absent from the simulations. Each simulation is ran for one simulated year. Six different horizontal grids are used in this study, which are provided in Table~\ref{tbl:table1}. All analyses exclude the first month of the simulations, and are computed on their native grids unless otherwise stated.

The horizontal hyper-viscosity operators $\nu$ vary with resolution after \cite{HETAL2019JAMES}, also provided in Table~\ref{tbl:table1}. The values of $\nu$ are a factor 2.5 greater for divergence damping and are not shown. $\Delta t_{phys}$ is chosen to scale with resolution, in proportion to the grid spacing,
\begin{equation}
\Delta t_{phys} = \Delta t_{phys,0} \times \frac{n_{e,0}}{n_e}~s,\label{eq:dt-scale}
\end{equation}
where $\Delta t_{phys,0}$ is taken to be the standard $1800$ s used in CAM-SE-CSLAM for the standard climate resolution, $n_{e,0} = 30$ (equivalent to an average equatorial grid spacing $\Delta x = 111.2$km). This scaling was chosen to avoid large time-truncation errors in a rising moist bubble test \citep[Appendix A in][]{HETAL2019JAMES}, and it is understood that this choice of $\Delta t_{phys}$ will likely lead to greater resolution sensitivity \citep{W2008TELLUS}. The convective time-scale in the deep convection scheme is fixed at 3600 s in all simulations.
 
 \section{Results}
 
  \begin{table*}
 \caption{Experimental design and global mean climatologies.}
 \centering
 \scriptsize
 \begin{tabular}{lcccccc}
   \hline
   Variable & $ne20$ & $ne30$ & $ne40$ & $ne60$ & $ne80$ & $ne120$ \\ 
   \hline
   $\Delta x$ (km) & 166.8 & 111.2 & 83.4 & 55.6 & 41.7 & 27.8 \\
   $\nu$ ($m^4/s$) & $1.5 \times 10^{15}$ & $4.0 \times 10^{14}$ & $1.5 \times 10^{14}$ & $4.0 \times 10^{13}$  & $1.5 \times 10^{13}$ & $4.0 \times 10^{12}$\\
    $\Delta t_{phys}$ (s) & 2700 & 1800 & 1350 & 900 & 675 & 450 \\
   Total Cloud Fraction & 0.844 & 0.835 & 0.824 & 0.810 & 0.804 & 0.800 \\ 
   Total Precipitable Water (mm) & 23.31& 23.01 & 22.62 & 22.25 & 21.93 & 21.72 \\
   Convective Precipitation (mm/day) & 1.91 & 1.83 & 1.68 & 1.47 & 1.29 & 1.08 \\
   Stratiform Precipitation (mm/day) & 1.26 & 1.42 & 1.60 & 1.85 & 2.05 & 2.22 \\      
 \hline
 \end{tabular}
 \label{tbl:table1}
 \end{table*}

Table~\ref{tbl:table1} provides globally averaged, climatological metrics for the CAM6 simulations which are typically published in CAM/CCM convergence studies. Total precipitable water, total cloud fraction and deep convective precipitation rate decreases, while stratiform precipitation increases, monotonically with resolution (also shown in Figure~\ref{fig:cam-history}). Resolution sensitivity in CAM6 is similar to all prior versions of the model. 

\subsection{Vertical Velocities and Resolution}

\begin{figure}
\begin{center}
\noindent\includegraphics[width=20pc,angle=0]{figs/temp_2pdf.pdf}\\
\end{center}
\caption{Probability density distribution of the upward vertical pressure velocities $\omega$ computed everywhere in the model from six-hourly output over the entirety of the year-long simulations. (a) Values on their native grid (solid) and values remapped to the $ne20$ grid (dotted), (b) values on their native grid, scaled to the $ne120$ resolution.}
\label{fig:2pdf}
\end{figure}

The PDF of negative, or upward vertical pressure velocities $\omega$ in the aqua-planets is shown in Figure~\ref{fig:2pdf}a. The magnitude of upward $\omega$ increases monotonically with resolution, with positive, or downward $\omega$ behaving similarly (not shown). This monotonic increase in the magnitude of $\omega$ is evident even after remapping all the model output to a common grid ($ne20$; dotted curves in Figure~\ref{fig:2pdf}a).

The PDF's may be scaled to the highest-resolution resolution grid through $P(\omega)_s = \alpha P (\omega / \alpha)$, where $\alpha$ is the scale factor from equation~\ref{eq:alpha}, and setting $\Delta x_0$ to the $ne120$ grid-spacing. Figure~\ref{fig:2pdf}b shows the scaled PDF's using $n=-1$ in equation~\ref{eq:alpha}. The scaled PDF's all collapse onto the high-resolution reference, indicating that the power-law with $n=-1$ explains to first-order the variation in vertical velocity with resolution in the aqua-planet simulations. 

Changes to the vertical velocity field can be further understood through decomposing the mass weighted vertical mean $\omega$ into upward and downward components,
\begin{equation}
\langle \omega \rangle =\langle f_{u} \rangle \, \langle \omega_{u} \rangle + \langle f_{d} \rangle \, \langle \omega_{d} \rangle, \label{eq:omega}
\end{equation}
where $\langle f_x \rangle$ and $\langle \omega_x \rangle$ refers to the vertical mass fraction $ \left( \frac{\int dp_x}{\int dp} \right)$ and the $x$ component of the mass weighted vertical mean of $\omega$ $ \left( \frac{\int \omega_x dp_x}{\int dp_x} \right)$, respectively, subscript $u$ refers to upward motion and $d$, downward motion.

The global mean, climatological components $\langle f_{u} \rangle \, \langle \omega_{u} \rangle$ and $\langle f_{d} \rangle \, \langle \omega_{d} \rangle$ are provided in Figure~\ref{fig:2panel}a,b for the aqua-planet simulations. The magnitude of both $\langle f_{u} \rangle \, \langle \omega_{u} \rangle$ and $\langle f_{d} \rangle \, \langle \omega_{d} \rangle$ increase monotonically with resolution, and are equal and opposite, which is a requirement of mass conservation in the model and a convenient check of the calculation. While $\langle f_{d} \rangle$ is about 25\% larger than $\langle f_{u} \rangle$ in all simulations, the vertical mass fractions varies by only few percent with resolution, and so the monotonic behavior of $\langle f_{x} \rangle \, \langle \omega_{x} \rangle$ with resolution is primarily from variations in $ \langle \omega_{x} \rangle$ (not shown).

\begin{figure}
\begin{center}
\noindent\includegraphics[width=20pc,angle=0]{figs/temp_diags_2panel.pdf}\\
\end{center}
\caption{Components of the climatological, global mean vertical pressure velocity, (a) $\langle f_{u} \rangle \langle \omega_{u} \rangle$ and (b) $\langle f_{d} \rangle \langle \omega_{d} \rangle$. Grey crosses are for the perturbed parameter ensemble runs.}
\label{fig:2panel}
\end{figure}

\subsection{Vertical Velocities and Convective Precipitation}

The large increase in magnitude of the upward and downward vertical velocities with resolution may be expected to impact the behavior of other model components. Curiously, there is an excellent negative correlation (Pearson's R-value = 0.99) between the global mean, climatological $\langle f_{d} \rangle \, \langle \omega_{d} \rangle$ and a measure of the activity of the \cite{ZM1995AO} deep convection scheme (referred to as the {\em{ZM scheme}} hereafter), global mean, climatological $FREQZM$ (Figure~\ref{fig:corr}). At any given grid-point and time-step, $FREQZM$ is a binary variable: one if the ZM scheme is active, zero if it is not. Time mean $FREQZM$ is therefore the fraction of the model time that the ZM scheme is triggered, i.e., dilute CAPE exceeds $\geq 70$ J/kg. In addition to the six simulations used in the convergence experiment, an ensemble of 24 additional simulations containing different model parameters (e.g., using the lower resolution physics grid) and across different resolutions are used, increasing confidence in the relationship between subsidence and convective activity.

\begin{figure}
\begin{center}
\noindent\includegraphics[width=20pc,angle=0]{figs/temp_diags_corr.pdf}\\
\end{center}
\caption{Scatter plot of global mean, climatological $\langle f_{d} \rangle \langle \omega_{d} \rangle$ and FREQZM, and the fitted linear regression which has a Pearson's R-value = 0.99. Grey crosses are for the perturbed parameter ensemble runs.}
\label{fig:corr}
\end{figure}

To further understand this relationship, a logistic regression between $\langle f_{d} \rangle \langle \omega_{d} \rangle$ and $FREQZM$ is performed for each grid column within each of the simulations. Logistic regression uses an iterative method to fit a continuous variable predictor, $x$ to a binary predictand $p$ through the exponential \citep{WILKSBOOK},
\begin{equation}
p = \frac{exp{[b_0 + b_1 x]}}{1 + exp{[b_0 + b_1 x]}}, \label{eq:eq6-3}
\end{equation}
where $b_0$ and $b_1$ are the shape parameters of the exponential. The predictor is then the instantaneous $\langle f_{d} \rangle \langle \omega_{d} \rangle$ of a grid column, and the predictand the binary $FREQZM$. The assumption is then that subsidence is the independent variable, which the authors believe is reasonable given subsiding regions are generally more stable, and so dilute CAPE is likely smaller compared with ascending regions. Grid column regressions that are statistically significant at the $95\%$ level using a log-likelihood test \citep{WILKSBOOK} are retained for analysis. Since the aqua-planets have zonally symmetric boundary conditions, there is a zonally varying structure in the goodness of fit (R-value) and parameter $b_1$ (hereafter referred to as the sensitivity parameter; Figure~\ref{fig:4zonal}a,b).

\begin{figure}
\begin{center}
\noindent\includegraphics[width=14pc,angle=0]{figs/temp_4zonal.pdf}\\
\end{center}
\caption{Zonal mean (a) R-values and (b) sensitivity parameter in the logistic regression, (d) time mean surface latent heat fluxes and (c) drizzle fraction. Colors are as in Figure~\ref{fig:2pdf}.}
\label{fig:4zonal}
\end{figure}

The zonal mean R-values indicate the greatest goodness of fit in the $\pm 10^{\circ}$ latitude band, hereafter referred to as the deep Tropics. Figure~\ref{fig:4zonal}c shows the climatological, zonal-mean latent heat flux in the simulations, which is expected to contribute positively to the component of dilute CAPE associated with the thermodynamic state of boundary layer parcels \citep{Z2002JGR}. In the deep Tropics, latent heat fluxes are small, and the sensitivity parameter is large and negative (Figure~\ref{fig:4zonal}b), consistent with the idea that subsiding motion actively depresses dilute CAPE and the activity of the ZM scheme in the simulations. The sensitivity parameter becomes less negative in the deep Tropics with resolution, likely due to the greater magnitude $\langle f_{d} \rangle \langle \omega_{d} \rangle$ with resolution, which requires a lower sensitivity parameter to predict the binary $FREQZM$. The R-values generally decrease with resolution indicating that there is degradation in the relationship with resolution.

Table~\ref{tbl:table2} shows the fractional contribution of the deep Tropics to the climatological, global mean change in convective precipitation with resolution. This is a reflection of changes in the partitioning of the Intertropical Convergence Zone (ITCZ) between convective and stratiform precipitation with resolution. The table indicates that a majority ($60-70 \%$) of the reduction in convective precipitation with resolution is from changes within the deep Tropics (except in going from $ne20$ to $ne30$, where deep Tropical convective precipitation rates increase with resolution due to an extremely wide double-ITCZ in the $ne20$ simulation that spans well outside of $\pm 10^{\circ}$ latitude). Expanding the latitudes of consideration marginally to $\pm 15^{\circ}$, roughly $75\%$ of the changes in convective precipitation with resolution occur in this region (again, ignoring $ne30-ne20$). Taken together, the region with the largest change in convective precipitation with resolution is also the region where the logistic regression indicates that subsiding motion is most skillful at depressing the activity of the convection scheme.

 \begin{table}
 \caption{Fractional contribution of latitude bands $\pm 10^{\circ}$ and $\pm 15^{\circ}$ to changes in global mean precipitation with resolution. The grid headers refer to differences with respect to the next lowest grid resolution, e.g., $ne30 = ne30-ne20$, $ne40=ne40-ne30$, etc..., and computed through remapping all data to the $ne20$ grid.}
 \centering
 \scriptsize
 \begin{tabular}{lcccccc}
   \hline
   Variable & $ne30$ & $ne40$ & $ne60$ & $ne80$ & $ne120$ \\ 
   \hline
   $\pm 10^{\circ}$ ($17.6\%$ of global area) \\
   Convective Precipitation & -0.58 & 0.62 & 0.66 & 0.72 & 0.70 \\
   Stratiform Precipitation & 0.55 & 0.63 & 0.69 & 0.67 & 0.41 \\ 
   \hline
   $\pm 15^{\circ}$ ($25.8\%$ of global area) \\
   Convective Precipitation & 0.22 & 0.75 & 0.73 & 0.79 & 0.72 \\
   Stratiform Precipitation & 0.46 & 0.64 & 0.71 & 0.70 & 0.49 \\      
 \hline
 \end{tabular}
 \label{tbl:table2}
 \end{table}

To estimate the dilute CAPE values associated with subsiding motion in the deep Tropics, temperature and moisture profiles are conditionally sampled depending on whether $\langle \omega \rangle$, the mass-weighted vertical integral of $\omega$, is positive or negative, indicating predominantly subsiding or ascending grid columns. The time mean temperature and moisture profiles of subsiding and ascending regions are then used to compute the dilute CAPE used in the ZM scheme, offline. Figure~\ref{fig:cape}a shows the dilute CAPE values associated with mean conditions for ascending, descending and all grid columns in the deep Tropics, with resolution. The ascending regions are associated with larger values of CAPE ($>180$ J/kg) relative to subsiding regions ($<110$ J/kg), and the CAPE value computed for mean conditions over the entire deep Tropics decrease monotonically with resolution, consistent with the reduction in convective activity with resolution. The space-time weights associated with ascending and descending grid columns in the deep Tropics vary drastically with resolution (Figure~\ref{fig:cape}b). The subsiding (ascending) space-time weights vary from $0.32$ ($0.68$) at $ne20$, monotonically increasing (decreasing) with resolution to $0.51$ ($0.49$) in $ne120$. The increasing occurrence of stable, subsiding grid columns with resolution results in a reduction in CAPE for the entire deep Tropics, verified by the similar CAPE values produced through taking the weighted sum of the ascending/descending CAPE values (grey crosses in Figure~\ref{fig:cape}a).

\begin{figure}
\begin{center}
\noindent\includegraphics[width=20pc,angle=0]{figs/temp_cape.pdf}\\
\end{center}
\caption{(a) time mean fraction of the deep tropics in the simulations with upward $\langle \omega \rangle$ (red) and downward $\langle \omega \rangle$ (blue), (b) CAPE computed from the mean temperature and moisture profiles of upward regions and downward regions. Black is for CAPE computed from the mean temperature and moisture profiles for the entire deep tropics, grey is the approximate discussed in the text.}
\label{fig:cape}
\end{figure}

Figure~\ref{fig:profiles} shows the climatological temperature and specific humidity profiles of subsiding grid cells in the deep Tropics, presented as anomalies from the mean temperature and specific humidity of the entire deep Tropics. The mean profiles of subsiding regions have an anomalous warming layer in the $600-800$ hPa layer and an anomalous moisture deficit throughout the entire column. This warming and drying patterns is consistent with the effects of subsidence, whose motion results in adiabatic warming and the advection of drier, upper-troposphere air downward. Both warming and drying the environment oppose the growth of dilute CAPE through reducing parcel buoyancy; warming the environment relative to the parcel temperature reduces parcel buoyancy \citep{Z2002JGR}, while mixing drier environmental air into a rising air parcel reduces the moisture available to warm the parcel through latent heating \citep{RB1992JAS}. 

\begin{figure}
\begin{center}
\noindent\includegraphics[width=20pc,angle=0]{figs/temp_profiles.pdf}\\
\end{center}
\caption{}
\label{fig:profiles}
\end{figure}

An experiment is carried out to unravel the contributions of warming or drying to dilute CAPE values of subsiding grid columns. Dilute CAPE is re-computed for descending/ascending regions, but using the mean specific humidity profile derived from the entire deep Tropics. Then the opposite is performed, computing dilute CAPE using the ascending/descending regions specific humidity but with the temperature profile from the entire deep Tropics.


%The dilute CAPE calculation is a modified form of the traditional CAPE, which is shown to be sensitive to the vertical velocity field in an analysis of a large eddy simulation \citep{SZ2018JCLIM}. A relationship between vertical velocities and CAPE falls out of the CAPE equation itself (F. Song, personal communication). The CAPE budget can be separated into two components \citep{Z2002JGR}; instability due to the thermodynamic state of parcels in the boundary layer and the instability generated through radiative cooling and advection of dry static energy and moisture by the environment, i.e., the resolved flow. The latter term is defined as, 
%\begin{equation}
%-R_d \int_{p_t}^{p_b} \frac{\partial T_{ve}}{\partial t} dlnp \label{eq:cape}
%\end{equation}
%where $T_{ve}$ is the virtual temperature of the environment, $R_d$ the gas constant for dry air and subscripts $b$ and $t$ refer to the parcel launch level (typically in the boundary layer) and the level of neutral buoyancy, respectively \citep{Z2002JGR}. Equation~\ref{eq:cape} shows that warming the environment reduces CAPE though reducing the buoyancy of lifted air parcels. The $T_{ve}$ budget contains a term; the vertical advection of potential energy, which simplifies to $\frac{w}{c_{pd}} \frac{\partial gz}{\partial z} = \frac{g}{c_{pd}} w$, and directly proportional to the vertical velocity $w$ by the factor $\frac{g}{c_{pd}}$, with $g$ the acceleration of gravity and $c_{pd}$ the specific heat capacity of dry air. In subsiding regions, $w$ is negative and opposes the generation of CAPE through adiabatic warming of the environment.

\subsection{Convection Biases Independent of Resolution}

Poleward of the deep tropics and within the subtropics, the logistic regression indicates that subsidence is a poor predictor of $FREQZM$ (Figure~\ref{fig:4zonal}a,b). The R-value decreases to a local minimum between $15^{\circ} - 20^{\circ}$ latitude, and the magnitude of the sensitivity parameter steeply declines. The local minimum in the R-value corresponds with a local maximum in the latent heat fluxes (Figure~\ref{fig:4zonal}c), indicating that the boundary layer is being driven unstable by large surface latent heat fluxes. The CAPE values in the $10^{\circ} - 15^{\circ}$ latitude region are likely to be small, since the ZM precipitation rate consists primarily of drizzle (Figure~\ref{fig:4zonal}d). The predominance of drizzle in this region is probably a result of the large subsiding motion in the subtropics (not shown) constraining CAPE from becoming too large. AGCMs are known to suffer from an excess drizzle bias in precisely this region \citep{D2006JCLIM}, and this analysis indicates that this bias is due to the use of a CAPE trigger function.

\subsection{Vertical Velocities and Stratiform Precipitation}

\begin{figure}[t]
\begin{center}
\noindent\includegraphics[width=20pc,angle=0]{figs/temp_pdecomp.pdf}\\
\end{center}
\caption{}
\label{fig:profiles}
\end{figure}

\ack 
Funding support for this work was in part provided by the U.S. Department of Energy Office of Science (DE-SC0019459) and the National Science Foundation (AGS1648629).

\bibliographystyle{wileyqj}
\bibliography{bib}
\end{document}
